\documentclass[10pt]{article}
\usepackage[total={170mm,230mm}]{geometry}

\usepackage{cmap}
\usepackage[utf8]{inputenc}
\usepackage[T2A]{fontenc}
\usepackage[russian]{babel}
\usepackage{hyphsubst}

\usepackage{graphicx}
\usepackage{xcolor}
\usepackage{amssymb}
\usepackage{amsfonts}
\usepackage{amsmath}
\usepackage{amsthm}
\usepackage{physics}
\usepackage{wrapfig}
\usepackage{cancel}
\usepackage{pdfpages}
\usepackage{hyperref}
\usepackage{caption}
\usepackage{subcaption}
% \usepackage{bibtex}

\title{Домашнее задание №6. Динамика гауссовского волнового пакета на фиксированном промежутке с использованием маскирующей функции}
\author{Александр Козлов}
\date{\today}

\begin{document}

\maketitle

\section*{Формулировка задания}

Рассматривается временное уравение Шрёдингера $i \partial_t \Psi = H \Psi$ с начальным условием $\Psi(k,t=0) = \exp(-k^2)$, где $k$ --- волновое число (в атомных единицах --- в которых, мы, собственно, и работаем --- тождественно импульсу). Необходимо воспроизвести свбодную (то есть $H=-\laplacian$ в атомарных единицах) динамику гауссовского волнового пакета на фиксированном промежутке с использованием маскирующей функции $M(x)$ и сравнить численное решение с аналитическим.

\section{Решения задачи с использованием маскировочной функции}

\subsection*{Сетка}
Прежде всего зададим равномерную сетку
\begin{equation}
	x_0 = -R,\; x_1 = x_0 + \delta,\; x_2 = x_0 + 2\delta,\; \ldots,\; x_k = x_0 + k\delta,\; \ldots,\; x_M = x_0 + M \delta = R
\end{equation}
с шагом $\delta = 2R/M$, где $M$~---~целое положительное число, а $R$~---~положительное действительное число.

\subsection*{Маскировочная функция}
В качестве маскировочной функции возьмём такую:
\begin{equation}
	M(x) = 
	\begin{cases}
		-\dfrac{2}{\delta_M^3}(x+R-\delta_M)^3 - \dfrac{3}{\delta_M^2} (x+R-\delta_M)^2 + 1, &x\in(-R,\; -R+\delta_M);\\
		1, &x\in(-R+\delta_M,\; R-\delta_M);\\
		\dfrac{2}{\delta_M^3}(x-R+\delta_M)^3 - \dfrac{3}{\delta_M^2} (x-R+\delta_M)^2 + 1, &x\in(R-\delta_M,\; R)
	\end{cases}
\end{equation}

\subsection*{Начальное условие}
Начальное условие задано в виде функции импульса, нормированной таким образом, что $\int \dd{k} \abs{\Psi(k,t=0)}^2 = \sqrt{\pi/2}$. Однако, мы планируем решать задачу в координатном представлении, поэтому следует получить начальное условие в координатном представлении, для чего делаем преобразование Фурье и получаем
\begin{equation}
	\Psi(x,t=0) = c_1 \int \dd{k} e^{ikx}\, \Psi(k,t=0) = c_1 \int \dd{k} e^{-k^2+ikx} = c_1 \sqrt{\pi}\, e^{-x^2/4},
\end{equation}
где $c_1$ --- нормировочный коэффициент. Выберем $c_1$ таким образом, чтобы волновая функция была нормирована на 1
\begin{equation}
	c_1^2 \pi \sqrt{2\pi} = 1,\quad c_1 = 2^{-1/4} \pi^{-3/4}.
\end{equation}
Отсюда следует, что начальное условие в координатном представлении имеет вид
\begin{equation}
	\Psi(x,\,t=0) = \dfrac{e^{-x^2/4}}{\sqrt[4]{2\pi}}.
\end{equation}

\subsection*{Аналитическое решение}
В ходе лекционных занятий было получено аналитическое решение задачи, оно имеет вид:
\begin{equation}
	\Psi(x,\,t) = \dfrac{e^{-x^2/4(1+it)}}{\sqrt[4]{2\pi}\sqrt{1+it}}.
\end{equation}

\subsection*{Итерационная последовательность волновых функций}
Для решения задачи строим итерационную последовательность волновых функций, обозначать которые будем через $\Psi^{(M)}_n(x\in[-R,\,R])$, где $n$ --- номер шага по времени. Волновая функция $\Psi^{(M)}_n(x)$, полученная предложенным алгоритмом решения, является приближением точного решения $\Psi(x,\, t=n\tau)$, где $\tau$ --- шаг по времени. Последовательность функций $\qty{\Psi^{(M)}_n(x)}_{n=0}^{N}$ формируется следующим образом:
\begin{equation}
	\begin{split}
		\Psi^{(M)}_0(x) &= M(x)\, \Psi(x,t=0),\\
		\Psi^{(M)}_1(x) &= M(x)\, e^{-i H \tau} \Psi^{(M)}_0(x),\\
		&\ldots\\
		\Psi^{(M)}_n(x) &= M(x)\, e^{-i H \tau} \Psi^{(M)}_{n-1}(x),\\
		&\ldots\\
		\Psi^{(M)}_N(x) &= M(x)\, e^{-i H \tau} \Psi^{(M)}_{N-1}(x).
	\end{split}
\end{equation}

Оператор эволюции $e^{-i H \tau}$, выбрав шаг по времени достаточно малым, можно расчитывать используя Паде-апрроксимацию.

\subsection*{Метод Рунге--Кутты 4-го порядка}
Для решения эволюционной задачи можно использовать и метод Рунге--Кутты 4-го порядка, если привести исходное временное уравнение Шрёдингера к эквивалентному уравнению
\begin{equation}
 \partial_t \Psi = f(x, \Psi),\quad
 f(x, \Psi) = i\dfrac{\Psi(x-\Delta x, t) - 2 \Psi(x, t) + \Psi(x+\Delta x, t)}{2 (\Delta x)^2}.
\end{equation}


\end{document}
